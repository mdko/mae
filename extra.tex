% Put in main presentation if needed
\begin{frame}{Referent Objects Approaches}
\footnotesize
J\&K \footcite{jones_backwards-compatible_1997}
    \begin{itemize}
       \item Maintain table of all known valid storage objects
       \item Map a ptr to a desciptor of the object into which it points
       \item Pad objects w/ extra byte
       \item check ptr arith and use, b/c result cannot refer to diff object from one from which it is originally derived
       \item object for which the ptr is valid is only determined by checking ptr itself, looking it up in object table
       \item Incomplete because cannot pad parameters, would change layout
       \item 11-12x overhead
    \end{itemize}

CRED \footcite{ruwase_practical_2004}
    \begin{itemize}
      \item More generic solution
      \item 2x overhead
    \end{itemize}
\vspace{0.2in}
\end{frame}

\begin{frame}{BBC}
\footnotesize
Baggy Bounds Checking (BBC) \footcite{akritidis_baggy_2009}
    \begin{itemize}
    \item Trades memory for performance, fastest object bounds checker
    \item Compact bounds representation and efficient way to look up object bounds
    \item Align base addresses to be multiple of padded size
    \item Replace splay tree with small lookup table
    \item 0.6x overhead on SPECINT 2000 benchmark
    \item Partition memory into slots with slot\_size bytes (table has entry per slot rather than per byte)
    \item Pad every object s.t. size is power of two, align base addresses to be multiple of their padded size
    \item Mark if OOB to prevent later dereference
    \item Handle OOB withint slot\_size/2 bytes from original object
    \end{itemize}
\vspace{0.2in}
\end{frame}

%PariCheck \footcite{younan_paricheck_2010} \note{like BBC, faster/lower overhead}

\begin{frame}{Other Spatial Approaches}
\begin{itemize}
    \item Purify \footcite{hastings_purify:_1991} (reveal during testing, object based, has temporal safety)
    \item Valgrind \footcite{nethercote_how_2007} (reveal during testing)
    %\item Qin SafeMem 2005
    %\item Venkataramani Memtracker 2007
\end{itemize}
\end{frame}

\begin{frame}{Intel MPX}
\footnotesize
  Intel MPX %\url{https://gcc.gnu.org/wiki/Intel%20MPX%20support%20in%20the%20GCC%20compiler}
    \begin{itemize}
     \item "new instructions that a compiler can insert to accelerate disjoint per-pointer metadata access and bounds checking"
     \item 4 new 128-bit bound registers
     \item BNDCL, BNDCU
     \item BNDSTC, BNDLDX
     \item BNDMK -- create bounds metadata for a pointer
     \item Extends func call conventions to include these bound registers
     \item Disjoint metadata for ptrs in memory
     \item Does not address temporal errors
     \item 'Incremental deployment' by checking if loaded pointer is same as redundantly-stored ptr in disjoint metadata structure => loss of comprehensiveness
     \item Metadata isn't updated on *integer stores* (b/c don't know to treat it as a ptr)
    \end{itemize}
\end{frame}

\begin{frame}{Alias Types}
  Alias types \footcite{smith_alias_2000}
    \begin{itemize}
        \item Problem: registers must be reused for data of different types constantly
        \item Aliasing constraints: describe shape of store; functions use to specify what they expect part of the store to look like
        \item Location and store polymorphism: dependence between pointer types and constraints, abstract away size/shape of store
        \item More expressive than linear: although aliasing constraints are linear, ptr values that flow through computation are not
        \item Constraint is mapping from locations to types; Pointer to location l has singleton type ptr(l)
    \end{itemize}
\end{frame}

\begin{frame}{More Object-Based Temporal Safety}
\begin{itemize}
\item Mark locations which were de-allocated in shadow memory space (i.e. track a few bits of state for each byte in memory, indicating if location is currently valid)
\item Detect access of de-allocated locations
\item Fails if pointer points again to re-allocated space
%\item \footcite{yong_protecting_2001}
\item Memcheck (10x slowdown)
\item Address Sanitizer \footcite{serebryany_addresssanitizer:_2012}
    \begin{itemize}
        \item Tripwire approach
        \item 73\% slowdown
        \item Can detect small-stride buffer overflows
    \end{itemize}
\end{itemize}
\end{frame}

\begin{frame}{Extended Static Checking}
Extended Static Checking (ESC) \footcite{detlefs_overview_1995}
    \begin{itemize}
      \item Use automatic theorem prover to detect index bounds in Modula-3
      \item Use info in annotations to assist
      \item Easier than C b/c no ptr arithmetic
    \end{itemize}
Also see: ESC/Java \footcite{flanagan_extended_2002}
\end{frame}

\begin{frame}{LCLint}
LCLint \footcite{larochelle_statically_2001}
\begin{itemize}
    \item Leverage LCLint, an annotation-assisted buffer detection tool
    \item Annotations that constrain possible values a reference contains before/after funcall
    \item Function pre/post-condition with: $\footnotesize{\texttt{requires, ensures, unique, returned, modifies, out}}$ clauses
    \item Assumptions are $\texttt{minSet, maxSet, minRead, maxRead}$
    \item Generates constraints at expression level, resolved w/ checking at statement level
    \item Heuristics to deal with loops nicely enough; neither sound nor complete
\end{itemize}
    Also see: ESC/Modula-3 \footcite{detlefs_overview_1995}, ESC/Java \footcite{flanagan_extended_2002}
    \vspace{0.2in}
\end{frame}

\begin{frame}{CSSV}
CSSV \footcite{dor_cssv:_2003}
\begin{itemize}
    \item Source-to-source translation
    \item Instruments program w/ additional variables describing string attrs
    \item Adds assert statements checking for unsafe string ops
    \item Statically analyze instr. version with \emph{integer analysis} to determine possible assertion failures
    \item Handles overlapping ptrs, etc.
    \item Disadv: \# vars in instr. quadratic in \# in orig.
\end{itemize}
\end{frame}

\begin{frame}{Early Use of Regions}
\footnotesize
RC \footcite{gay_language_2001}
    \vspace{-0.1in}
   \begin{itemize}
     \item Region-support for C, focus on preventing dangling pointers % aliasing!
     \item No restrictions on region lifetimes, pointers can point anywhere
     \item Maintain a reference count of external pointers pointing to objects in region % deleteregion fails if count is zero
     \item Type system + annotations ($\texttt{sameregion}, \texttt{traditional}, \texttt{parentptr}$) to help compiler remove annotation operations % no updating associated region % enforced by runtime checks
     \item Annotations ($\texttt{sameregion}, \texttt{traditional}, \texttt{parentptr}$) to prevent updating associated region % enforced by runtime checks
     \item Restrict C by disallowing arbitrary integer-to-pointer casts
     \item Break cyclic data structures before deleting regions
     \item \alert{No} memory safety guarantee
     \item 27\%-11\% overhead
   \end{itemize}
\end{frame}

\begin{frame}{More Early Use of Regions}
Reaps \footcite{berger_reconsidering_2002}
    \vspace{-0.1in}
   \begin{itemize}
     \item Combine regions and heaps into ``reaps''; regions until $\texttt{reapFree}$ deallocates individual object
     \item Place freed object on a heap; subsequent allocations use heap until exhausted
     \item Use Lea allocator generally, unless you need fast regions (i.e. reaps) % what their study found
   \end{itemize}

Control-C \footcite{kowshik_ensuring_2002}
    \vspace{-0.1in}
    \begin{itemize}
        \item Single region active at any given time
        \item Pointer value containing region address must be provably dead before \texttt{rfree}
        \item Automatic Pool Allocation and Type Homogeneity \footcite{dhurjati_memory_2003} for safe dangling pointers \note{objects represented by the same points-to graph node are allocated in a common pool; can individually deallocate}
    \end{itemize}
\end{frame}

\begin{frame}{CETS}
  CETS
  \begin{itemize}
    \item Extends SoftBound
    \item Validity bit stored in global dictionary
    \item Each objects gets unique id as key to dictionary, pointers associated with id
    \item enforces full safety
    \item 48\% for temporal, 116\% overhead for full (Softbound-CETS) on SPEC
    \item incompatiblity issues of SoftBound too
   \end{itemize}
\end{frame}

\begin{frame}{Softbound-CETS}
\begin{itemize}
  \item In absence of wrappers, code will not experience vilations as long as external libaries d no return ptrs or update ptrs in memory
  \item Benefits from LLVM maintaing ptr info from src in IR: large \# of checks can be eliminated statically using check elim opt
  \item 76\% overhead on avg on SPEC benchmarks
\end{itemize}
\end{frame}

\begin{frame}{Memsafe}
\footnotesize
    Memsafe \footcite{simpson_memsafe:_2013}
    \begin{itemize}
      \item Use spatial data for temporal safety by setting bounds of deallocated ptr and all its aliased ptrs in metadata space to an invalid value;
      \item Subsequent deference of such a deallocated ptr will raise an exception as bounds metadata is invalid
      %\item Uses SoftBound approach of maintaining disjoint ptr-based metadata
      \item Also stores some object-based metadata in global database, looking up only when ptr-based meatadata is insufficient for proving temporal safety
      %\item Insert assignments of invalid ptr after free
      \item Free(p) => p=invalid (i.e. addr\_p->base=1 addr\_p->bound=0)
      \item Spatial safety checks involving the base/bound of invalid ptr are guaranteed to always report a safety violation
      \item Inserts assignments of invalid ptr at end of a procedure for each local variable whose address is taken (so ptr to stack object can't escape)
      \item Models in-memory ptr assignments as epxlicit assignments using alias analysis and $\phi$-like ssa extension call $\rho2$-function
    \end{itemize}
\end{frame}

% Regions example in Cyclone
\begin{frame}[fragile]{Regions Example}
\footnotesize{
  \begin{columns}[T]
    \begin{column}{0.48\textwidth}
\begin{lstlisting}[language=C,mathescape,basicstyle={\scriptsize\ttfamily}]
// Example prototypes
char?$\rho_1$ strcpy<$\rho_1$,$\rho_2$>(char?$\rho_1$ d,
  const char ?$\rho_2$ s);
char?$\rho_H$ strdup<$\rho$>(const char?$\rho$ s);
char?$\rho$ strlen<$\rho$>(const char?$\rho$ s);

// Dangling pointer prevention
int *$\rho_L$ p;
L:{ int x = 0;
    p = &x;
  }
*p = 42;
\end{lstlisting}
    \end{column}
%
%// Region polymorphism
%L:{ char buf[20];
%    strcpy<$\rho_L$,$\rho_H$>(buf, "a heap pointer"); }

    \begin{column}{0.48\textwidth}
\begin{lstlisting}[language=C,mathescape,basicstyle={\scriptsize\ttfamily}]
// Polymorphic Recursion
void fact<$\rho$>(int*$\rho$ result, int n) {
 L: { int x = 1;
      if (n > 1) fact<$\rho_L$>(\&x,n-1);
      *result = x*n; }
}
int g = 0;
int main() { fact<$\rho_H$>(\&g, 6); return g; }
\end{lstlisting}
    \end{column}
    \end{columns}
}
%
%// Type definitions
%struct Lst<$\rho_1$, $\rho_2$> {
%    int*$\rho_1$ hd;
%    struct Lst<$\rho_1$, $\rho_2$> *$\rho_2$ tl;
%};
\end{frame}

\begin{frame}{Quasi-Linear Types}
\footnotesize
  Quasi-linear types \footcite{kobayashi_quasi-linear_1999} % relaxtes linear type strong condition
    \begin{itemize}
        \item Relax linear type strong condition
        \item Distinguish between consumed values vs those that may be returned
        \item Use $\kappa$ to control how often a variable of type $\tau^{\kappa}$ is used
        \item $\kappa = \delta$: accessed many times \emph{locally}, cannot be returned
            \begin{itemize}
                \item 0: not used at all
                \item 1: value accessed at most once
                \item $\omega$: accessed arbitrary number of times
                \item $\delta$: accessed many times \emph{locally}, cannot be returned with result
            \end{itemize}
        \item Quasi-linear value (1) accessed as $\delta$ and then strictly as linear
        \item Inspired $\textsc{paclang}$
    \end{itemize}
\end{frame}

\begin{frame}{Other Linear Types}
\footnotesize
  Vault \footcite{deline_enforcing_2001}
    \vspace{-0.09in}
    \begin{itemize}
        \item Keys associate static capabilities with run-time resources %, held-key set, type guards
        \item Functions annotated with effect clause (pre- and post-conditions on held-key set contents)
        \item Freed regions before leaving scope
        \item Types enforce code must free a region
        \item static enforcement of various protocols
        \item Restrict aliasing, tracks fine-grained effects (requires more annotations)
        \item Windows 2000 locking errors, IRP ownership model % corresponds to tracked types
    \end{itemize}
\end{frame}

\begin{frame}{Ordered Types}
  Ordered types for memory layout \footcite{petersen_type_2003}
    \vspace{-0.09in}
    \begin{itemize}
        \item Restrict linear types (remove exchange property)
        \item Variables cannot change position $\Rightarrow$ locations in memory
        \item ``Orderly lambda calculus'' for size-preserving memory operations
        \item Coercions to manipulate ordered variables in frontier (combine/split to treat as different types)
    \end{itemize}
\end{frame}

\begin{frame}{Typestate}
  Typestate \footcite{strom_typestate:_1986}
    \begin{itemize}
        \item Avoid nonsensical execution sequences statically (using uninitialized value)
        \item Typestate is static invariant of each variable name at program point % i.e. need to merge control-paths, regardless of path to get to point
        \item Define a lattice of states and typestate transition system between them
        \item Linear types help because of restricting pointer assignment (1-1 mapping between variable names and run-time objects)
    \end{itemize}
\end{frame}

\begin{frame}{Some Practical Affine Types}
  Alms \footcite{tov_practical_2011}
    \begin{itemize}
        \item Practical and general purpose
        \item Affine types: a \emph{weakening} of linear types: can drop but not duplicate
        \item Affine capabilities: separate a read-only reference to array from an affine writeable reference
        \item Define $^{a}\lambda_{ms}$ (based on System $F^{\omega}_{<:}$) and proof of soundness \note{types are maintained during evaluation, assuming no divergence}
        \item Implemented a capability-based interface to Berkeley sockets
        \item Basis of Rust's type system
    \end{itemize}
\end{frame}

\begin{frame}{Low-Level Liquid Types}
Low-Level Liquid Types (LTLL) \footcite{rondon_low-level_2010}
\begin{itemize}
    \item Refinement types where predicates are conjunctions over qualifiers
    \item Functions qualified over locations they operate on
    \item Deal with collections using \emph{location folding} for checking out a copy to do strong updates on
    \item Tries to deal with lack of types, mutation, unbounded collections that make type-based mechanisms difficult
\end{itemize}
\end{frame}

\begin{frame}{Cyclone}
Cyclone \footcite{jim_cyclone:_2002}: \note{offshoot of TAL/popcorn}
\begin{itemize}
    \item Annotations for non-array vs array pointers (can specify size) % (never-null, fat, unitialized warning via control-flow analysis) % TODO look up again, split this
    \item Tagged unions and automatic tag injection
    %\item e.g. instead of var-args, know exactly what's stored in thing callee passed $\texttt{printf}$
    %\item Can specify number of elements pointer points to \note{deputy is more general}
    %\item Arrays and strings converted automatically to fat
    %\item Statically validate and remove non-array fat pointers
    \item Need user annotations more than other approaches
    \item 40 percent runtime overhead
    \item Uses regions + automatic memory management for temporal safety (free is a no-op) (see nice example)
    \item Never null don't need checks, use @; push back null checks from uses to their sources
    \item Restrict arithmetic on regular pointers
    %\item parametric polymorphism, subtyping, static analysis to check for safety, adding run-time checks
\end{itemize}
\end{frame}

\begin{frame}{All Cyclone's Abstract Syntax}
\scriptsize{
\begin{alignat*}{2}
\text{kinds}\quad &\kappa &&::= T \altm R
\\
\text{type and region vars}\quad &\alpha,\rho &&
\\
\text{region sets}\quad &\epsilon &&::= \emptyset \altm \alpha_n \epsilon_1 \cup \epsilon_2
\\
\text{region constraints} \quad &\gamma &&::= \emptyset \altm \gamma, \epsilon <: \rho
\\
\text{constructors} \quad &\tau &&::= \alpha \altm \text{int} \altm \tau_1 \xrightarrow[]{\epsilon} \tau_2 \altm \tau_1 \times \tau_2 \altm \tau \ast \rho \altm
\\
~ & ~ && \qquad \text{handle}(\rho) \altm \forall\alpha:\kappa \triangleright \gamma.\tau \altm \exists \alpha:\kappa \triangleright \gamma.\tau
\\
\text{expressions} \quad &e &&::= x_\rho \altm v \altm e\ \langle\tau\rangle \altm (e_1,e_2) \altm e.i \altm \ast e \altm \textbf{rnew}(e_1)e_2 \altm
\\
~ & ~ && \qquad e_1(e_2) \altm \&e \altm e_1 = e_2 \altm \textbf{pack} [\tau_1, e]\textbf{ as } \tau_1
\\
\text{values} \qquad &v &&::= i \altm f \altm \&p \altm \textbf{region}(\rho) \altm (v_1, v_2) \altm \textbf{pack}[\tau_1,v]\textbf{ as } \tau_2
\\
\text{paths} \qquad &p &&::= x_\rho \altm p.i
\\
\text{functions} \qquad &f &&::= \rho:(\tau_1\ x_{\rho}) \xrightarrow[]{\epsilon} \tau_2 = \{s\} \altm \Lambda\alpha:\kappa \triangleright \gamma.f
\\
\text{statements} \qquad &s &&::= e \altm \textbf{return } e \altm s_1;s_2 \altm \textbf{if } (e)\ s_1 \textbf{ else } s_2 \altm \textbf{while } (e)\ s \altm
\\
~ & ~ && \qquad \rho:\{\tau\ x_\rho = e;\ s\} \altm \textbf{region}\langle\rho\rangle\ x_{\rho}\ s \altm \rho:\{\textbf{open}[\alpha,x_{\rho}] = e;\ s\} \altm s\ \textbf{pop}[\rho]
\end{alignat*}
    }
\end{frame}

\begin{frame}{Fail-Safe C}
\begin{itemize}
\item Safe implementation of ANSI-C, handles casts very well
\item Extend fat pointers to also contain
    \begin{itemize}
        \item 'Cast' flag embedded in two-word representation: enables optimization by not needing to do a long memory access via reading header information elsewhere
        \item 'Virtual Offsets' instead of real memory address offsets
    \end{itemize}
\item Fat Integer: integer large enough to hold any pointer value
\item Virtual offset: corresponds to program-visible size
\item Virtual size: real size of equivalent data type in native C impl.
\end{itemize}
\end{frame}

\begin{frame}{Fat Pointers}
Cuckoo \footcite{west_cuckoo:_2005}
      \begin{itemize}
       \item Store array size in memory before array dimensions' first element
       \item Name of an array is pointer to an \emph{array}, not first object
       \item Type system for preventing assignment of automatic objects into longer-lifetime pointers
       \item Wrap dynamic memory allocation (type homogeneous pool-based) % will explain later
       \item Forbid addition and subtraction expressions including pointer operands
       \item Compile-time checks if array bounds are constants, otherwise run-time checks
      \end{itemize}
\end{frame}

\begin{frame}{Hardware and Other Support for Spatial Safety}
\begin{itemize}
    \item Hardbound \footcite{devietti_hardbound:_2008}
    \item Stackguard \footcite{cowan_stackguard:_1998} (inserts canaries)
    \item Light-weight Bounds Checking \footcite{hasabnis_light-weight_2012} (guard zones with good performance)
    \item Manual MM to high-level languages \footcite{kedia_simple_2017}
    \item PCC \footcite{necula_proof-carrying_1997}
    \item BitC \footcite{shapiro_origins_2008} (a retrospective on BitC, what they were looking for (not too interesting))
    \item Intel MPX %: \verbatim{https://gcc.gnu.org/wiki/Intel%20MPX%20support%20in%20the%20GCC%20compiler}
    \item Watchdog: Nagarakatte 2012
    \item (Hybrid): WatchdogLite
\end{itemize}
\end{frame}


\begin{frame}{Rust OS}
\footnotesize
Rust OS \footcite{levy_ownership_2015}
    \begin{itemize}
        \item Ownership \emph{hinders} resource sharing
        \item AMM not optimized for HW resources/device drivers
        \item Closures' req for dynamic memory is bad for embedded systems
        \item Many resources not dynamically allocated
        \item Mutably borrow static resources
        \item Embedded systems typically have one primary execution thread, so aliasing in same thread is okay
        \item Rust doesn't allow mutable aliasing, so extend type system with \alert{execution contexts}
        \item Type records thread of value's owner: allow multiple borrows of value within same thread, not across threads
    \end{itemize}
\end{frame}

\begin{frame}{No Time}
\begin{itemize}
  \item Mudflap: Eigler 2003
  \item Criswell: Secure Virt Arch 2007
  \item PariCheck: Younan: 2010
\end{itemize}
\end{frame}

